\documentclass[12pt]{article}
\usepackage{esqu1}
\pagestyle{fancy}

\lhead{Brandon Lin}
\chead{Lambda Calculus}
\rhead{}

\begin{document}
	\begin{center}
		\section*{Lambda Calculus -- Notes}
	\end{center}
	\textbf{Lambda calculus} is an abstract way of representing a programming language, commonly known as the \textit{smallest universal programming language} there is. The lambda calculus is equivalent to a single-tape Turing machine, where all abstractions are composed of functions. The way function are representing are through the syntax: \[ \text{function} \coloneqq \lambda <argument>.<expression>\]
	where the argument and expression are composed of placeholder variables. For example, the identity function $f(x) = x$ can be represented in the lambda calculus as \[ (\lambda x.x)\] We can also apply these functions to particular arguments, for example, the identity function evaluated at $y$ would be \[ (\lambda x.x)y\] Notice that when functions are being evaluated, the value gets ``substituted'' into the function. For this reason, we use the notation $[y/x]$ to denote that $y$ is being substituted in for $x$: \[ (\lambda x.x)y \to [y/x]x \to y\]
\end{document}
