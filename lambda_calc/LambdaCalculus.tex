\documentclass[12pt]{article}
\usepackage{esqu1}
\pagestyle{fancy}

\lhead{Brandon Lin}
\chead{Lambda Calculus}
\rhead{}

\begin{document}
	\begin{center}
		\section*{Lambda Calculus -- Notes}
	\end{center}
	\textbf{Lambda calculus} is an abstract way of representing a programming language, commonly known as the \textit{smallest universal programming language} there is. The lambda calculus is equivalent to a single-tape Turing machine, where all abstractions are composed of functions. The way function are representing are through the syntax: \[ \text{function} \coloneqq \lambda <argument>.<expression>\]
	where the argument and expression are composed of placeholder variables. For example, the identity function $f(x) = x$ can be represented in the lambda calculus as \[ (\lambda x.x)\] We can also apply these functions to particular arguments, for example, the identity function evaluated at $y$ would be \[ (\lambda x.x)y\] Notice that when functions are being evaluated, the value gets ``substituted'' into the function. For this reason, we use the notation $[y/x]$ to denote that $y$ is being substituted in for $x$: \[ (\lambda x.x)y \to [y/x]x \to y\]

	\subsection*{Free vs. Bound Variables}
	\textbf{Bound variables} are variables that are being bound by a $\lambda$ operator. For a variable to be bound, it must be the argument of a function. \textbf{Free variables} are variables that do not have this property. For example, in $(\lambda x.x)y$, $x$ is a bound variable, and $y$ is a free variable. 

	There are some cases in which a variable can both be bound and free; freedom is always relative to an expression. For example, in the expression $(\lambda x.xy)(\lambda y.y)$, $y$ is both a bound variable (in the function $\lambda y.y$) and a free variable (relative to the entire expression). 

	\subsection*{Reduction}
	Our ultimate goal is to take these expressions and have ways to reduce them into smaller pieces. There are three main ways:
	\subsubsection*{$\alpha$-reduction}
	Note that the exact letters used in a particular function don't matter; in fact, we could have used $z$ instead in the identity function: $\lambda z.z$. Substituting letters like this is known as \textbf{$\alpha$-reduction}, as particular variable names don't matter as long as you change it throughout an entire abstraction. 

	However, we must worry about \textbf{name capture}, changing the scope of a variable. The two functions $\lambda y.\lambda z.y$ and $\lambda y.\lambda y.y$ demonstrate this. The first is a function that returns a constant mapping function to the variable $y$. The second returns the identity function. We must be careful when we are performing $\alpha$-reduction so as to not cause this from happening. 

	\subsubsection*{$\beta$-reduction}
	\textbf{$\beta$-reduction} is very straightforward; it is when you evaluate a function at an argument expression. \[ (\lambda x.M)z \to M[z/x] \]

	\subsection*{Arithmetic}
	We can do arithmetic using lambda calculus. To do this, we define the function $\lambda s.(\lambda x.x)$ (the function that returns the identity function) to be zero. Then, we can define the rest of the natural numbers as follows: 1 is the function that returns $x \to f(x)$, 2 is the function that returns $x \to f(f(x))$, and so on.
	\[ 
	 	\begin{aligned}
	 		1 &\coloneqq \lambda s.(\lambda z.s(z)) \\
	 		2 &\coloneqq \lambda s.(\lambda z.s(s(z))) \\
	 		3 &\coloneqq \lambda s.(\lambda z.s(s(s(z))))
	 	\end{aligned}
	 \] 
	 From here, we can define a few operators that take on lambda function forms and can be interpreted in a clean way:
	 \begin{itemize}
	 	\item The \textbf{successor function} take an argument $x$ and outputs $x+1$: \[ \text{succ} \equiv \lambda n.\lambda f.\lambda x.f(nfx)\]
	 	The intuition behind this function is that given a lambda representation of an integer $m$, which is a function returning a function composed $m$ times, $n$ in the successor function is the dummy variable for this input. But note that $nfx$ will return $f^n(x)$, because if $n = \lambda s.\lambda z.s^n(z)$, then \[ nfx = (\lambda s.\lambda z.s^n(z))fx = (\lambda z.f^n(z))x = f^n(x) \] Thus, from the extra $f$ composition in the successor function, it will output $f^{n+1}(x)$.
	 	\item The \textbf{addition function} adds two numbers, taking advantage of the fact that $f^{m+n}(x) = f^m(f^n(x))$: 
	 	\[ \text{plus} \equiv \lambda m.\lambda n.\lambda f.\lambda x.mf(nfx)\] Recognize that this has a similar form as the successor function, except performed $m$ times.
	 \end{itemize}
\end{document}
